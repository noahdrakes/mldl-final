\documentclass[twocolumn]{article}

\usepackage{amsmath}
\usepackage{graphicx}
\usepackage{xcolor}
\usepackage[utf8]{inputenc}

\newcommand{\todo}[1]{\textcolor{red}{#1}}

\title{CS482/682 Final Project Report Group \todo{XX}\\
	\large \todo{your project title here}}

\author{Noah Drakes - ndrakes1}

\date{}

\begin{document}
	
	\maketitle
	
	% do not write an abstract
	\section{Problem Statement}
	% \paragraph{}
	
	The goal of this study is to replicate and fine-tune a currently existing anomaly detection modal in a semi-supervised fashion that utilizes multimodality, 
	video and audio inputs, in order to predict the presence of violent/anomalous events. Examples of anomalous 
	events would be a car crash happening on a busy streat or a fight breaking out in a subway. With the rise of advanced
	surveillance systems, there has been a push toward incorporating ML algorithims in the video security domain. 
	By utiling robust anomaly detection algorithims in computer vision, security teams can quickly and autonomously identify 
	suspicous activity in crowded areas and can help prevent anomalous events from escalating.

	In many current anomalous detection algorithims, only one modality (video) is processed to predict the occurence of irreugular events. 
	However, by adjoining the auditory modality, models can benefit from richer contextual understanding of scenes that are revealed 
	in audio samples such as screams, crashes, or other loud sounds. Furthermore, in cases in which both modalities 
	are low resolution (ie. noisy, compressed audio or blurry video), both modalities can help improve prediction accuracy.


	% \paragraph{Related Work} \todo{e.g. previous supervised approaches, other unsupervised methods etc}
	
	\section{Summary of Dataset}
	There are two datasets that are being considered for this project. The first being \textbf{XD-Violence} which
	comprises of 4754 untrimmed videos obtained form YouTube videos that are divided into with corresponding audio signals and weak labels (violent/normal).
	This dataset has gained popularity with research focusing on anomaly detection. We will have to do some preprocessing
	to shrink the video length and downsample the video resolution (x6 or x8 maybe). Videos are a very high-dimensional input and could 
	easily increase model complexity and training time. Another dataset being considered is the \textbf{UCF-Crime} comprising of 1900
	untrimmed videos of 13 realistic anomalous events, such as burglary, robbery, fighting, and so. 

	\section{Related Papers}
	The first reference, "Learning Multimodal Violence Detection under Weak Supervision", uses the XD-Violence dataset to detect Violence
	by fusing video and audio modalities and using HL-Net architecture to capture short term and long term temporal information [1]. A 3D CNN 
	is used for video feature extraction and VGGish (1D CNN) is used as audio feature extractor. HL-NET utilizes a local and global encoder 
	to capture short-term patterns (1 - 2s) and long term dependencies across the entire video, respectively. The output of the model is a 
	violence score based on individual snippets of video and audio which can be used for scene restricted violence prediction or video classification
	by max pooling or averaging. 


	\paragraph{Dataset} \todo{BRIEFLY describe the dataset and any pre-processing you may use that is special (e.g. downsample all images by x8 to enable colaboratory training etc.)}
	
	\paragraph{Setup, Training and Evaluation} \todo{IMPORTANT: you can change this paragraph to better fit your project. Questions to answer: What architecture did you chose and why? How did you accomplish un- or self-supervised learning? What tweaks were necessary to make this happen, e.g. custom layer designs, auxiliary tasks, etc.?}
	
	
	
	\section{Results}
	
	
	\section{References} 
	\begin{thebibliography}{9}
		\bibitem{texbook}
		Shaoyuan Xu, Qi Jin, Yueming Liu, Kai Wang, and Tianqiang Ruan. "Not Only Look, but also Listen: Learning Multimodal Violence Detection under Weak Supervision." *Proceedings of the European Conference on Computer Vision (ECCV)*, 2020.

		\end{thebibliography}

\end{document}